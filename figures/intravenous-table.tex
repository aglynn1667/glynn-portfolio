\jtable[caption=Intravenous or highly bioavailable oral antimicrobial treatment of common microorganisms causing prosthetic joint infection (B-III unless otherwise stated in text)., label=iv-treatment, maxwidth=\textwidth, doinside=\metafont, tablefont=\metafont]{
\tnote[]{Abbreviations: bid, twice daily; IV, intravenous; PJI, prosthetic joint infection; q, every; PO, per oral; qid, 4 times daily.}
\tnote[a]{Antimicrobial dosage needs to be adjusted based on patients' renal and hepatic function. Antimicrobials should be chosen based on in vitro susceptibility as well as patient drug allergies, intolerances, and potential drug interactions or contraindications to a specific antimicrobial. Clinical and laboratory monitoring for efficacy and safety should occur based on prior IDSA guidelines [6]. The possibility of prolonged QTc interval and tendinopathy should be discussed and monitored when using fluoroquinolones. The possibility of \emph{Clostridium difficile} colitis should also be discussed when using any antimicrobial.}
\tnote[b]{Flucloxacillin may be used in Europe. Oxacillin can also be substituted.}
\tnote[c]{There was not a consensus on the use of ceftriaxone for methicillin-susceptible staphylococci (see text).}
\tnote[d]{Target troughs for vancomycin should be chosen with the guidance of a local infectious disease physician based on the pathogen, its in vitro susceptibility, and the use of rifampin or local vancomycin therapy. Recent guidelines [<span class="xrefLink" id="jumplink-CIS803C155"></span>155, <span class="xrefLink" id="jumplink-CIS803C164"></span>164] for the treatment of methicillin-resistant \emph{Staphylococcus aureus} (MRSA) infections have been published. (These guidelines suggest that dosing of vancomycin be considered to achieve a vancomycin trough at steady state of 15 to 20. Although this may be appropriate for MRSA PJI treated without rifampin or without the use of local vancomycin spacer, it is unknown if these higher trough concentrations are necessary when rifampin or vancomcyin impregnated spacers are utilized. Trough concentrations of at least 10 may be appropriate in this situation. It is also unknown if treatment of oxacillin-resistant, coagulase-negative staphylococci require vancomycin dosing to achieve these higher vancomycin levels.)}
\tnote[e]{Other antipseudomonal carbapenems can be utilized as well.}
}{XXXX}{
    \toprule
    Microorganism & Preferred Treatment\tmark[a] & Alternative Treatment\tmark[a] & Comments\\
    \midrule
    \multirow{5}{*}{Staphylococci, oxacillin-susceptible} & Nafcillin\tmark[b] sodium 1.5–2 g IV q4-6 h  & Vancomycin IV 15 mg/kg q12 h  \multirow{5}{*}{See recommended use of rifampin as a companion drug for rifampin-susceptible PJI treated with debridement and retention or 1-stage exchange in text}\\
    or  & or  \\
    Cefazolin 1–2 g IV q8 h  & Daptomycin 6 mg/kg IV q 24 h  \\
    or  & or  \\
    Ceftriaxone\tmark[c] 1–2 g IV q24 h  & Linezolid 600 mg PO/IV every 12 h  \\
    Staphylococci, oxacillin-resistant  & Vancomycin\tmark[d] IV 15 mg/kg q12 h  & Daptomycin 6 mg/kg IV q24 horLinezolid 600 mg PO/IV q12 h  & See recommended use of rifampin as a companion drug for rifampin-susceptible PJI treated with debridement and retention or 1-stage exchange in text  \\
    \emph{Enterococcus} spp, penicillin-susceptible  & Penicillin G 20–24 million units IV q24 h continuously or in 6 divided doses \newline or\newline Ampicillin sodium 12 g IV q24 h continuously or in 6 divided doses  & Vancomycin 15 mg/kg IV q12 h \newline \newline or \newline Daptomycin 6 mg/kg IV q24 h  & 4–6 wk. Aminoglycoside optional \newline \newline Vancomycin should be used only in case of penicillin allergy  \\
      &   & or  &   \\
      &   & Linezolid 600 mg PO or \newline  IV q12 h  &   \\
    \emph{Enterococcus} spp, penicillin-resistant  & Vancomycin 15 mg/kg IV q12 h  & Linezolid 600 mg PO or \newline \newline IV q12 h \newline \newline or \newline Daptomycin 6 mg IV q24 h   & 4–6 wk. Addition of aminoglycoside optional  \\
    \emph{Pseudomonas aeruginosa}  & Cefepime 2 g IV q12 h  & Ciprofloxacin 750 mg PO bid  & 4–6 wk  \\
      & or  & or 400 mg IV q12 h  & Addition of aminoglycoside optional  \\
      & Meropenem\tmark[e] 1 g IV q8 h  & or  \multirow{2}{*}{Use of 2 active drugs could be considered based on clinical circumstance of patient. If aminoglycoside in spacer, and organism aminoglycoside susceptible than double coverage being provided with recommended IV or oral monotherapy}\\
      &   & Ceftazidime 2 g IV q8 h  \\
    \emph{Enterobacter} spp 
    Cefepime 2 g IV q12 h \newline or \newline Ertapenem 1 g IV q24 h  Ciprofloxacin 750 mg PO \newline  or 400 mg IV q12 h  4–6 wk.  \\
    Enterobacteriaceae  IV $\beta$-lactam based on in vitro susceptibilities or \newline Ciprofloxacin 750 mg PO bid    4–6 wk  \\
    $\beta$-hemolytic streptococci  Penicillin G 20–24 million units IV q24 h continuously or in 6 divided doses \newline or \newline Ceftriaxone 2 g IV q24 h  Vancomycin 15 mg/kg IV q12 h  4–6 wk \newline Vancomycin only in case of allergy  \\
    \emph{Propionibacterium acnes}  Penicillin G 20 million units IV q24 h continuously or in 6 divided doses \newline or \newline Ceftriaxone 2 g IV q24 h  Clindamycin 600–900 mg IV q8 h or clindamycin 300–450 mg PO qid \newline or \newline Vancomycin 15 mg/kg IV q12 h  4–6 wk \newline \newline Vancomycin only in case of allergy  \\
}